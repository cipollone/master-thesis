\chapter{AtariEyes package}

\label{ch:atarieyes}

In this Chapter, we'll discuss the software realized in this thesis, called
``AtariEyes''. Its purpose is to implement the ideas that have been presented
until this point. Every experiment of Chapter~\ref{ch:experiments} has been
obtained with this software.

Apart from the important aspect of realizing the models we've defined, this
software has some interesting qualities:
\begin{description}
	\item [Clarity] Because it includes a complete documentation of methods
		and structures.
	\item [Efficiency] Thanks to an heavy use of parallel computing libraries
		for GPU acceleration.
	\item [User friendly] The extensive command line interface allows to
		experiment with the package as it is, or, thanks to its modular design,
		individual structures can be reused in future developments.
\end{description}

Regarding the general functionality of the package, through the commands
provided, the user can:
\begin{itemize}
	\item Choose any \emph{environment} from the Atari~2600 collection.
	\item Train a Deep Reinforcement Learning \emph{agent}. The algorithm is
		Double DQN and the agent's model can be either the original Atari model
		(Section~\ref{sec:model-atari}) or the restricted agent
		(Section~\ref{sec:model-atari-rb}).
	\item Train the \emph{feature extractor}. This implements the whole model
		presented in Chapter~\ref{ch:fluents}.
	\item \emph{Play}, \emph{visualize} and \emph{record} any of these
		agents while they interact with the environment.
\end{itemize}

This chapter contains two sections: Section~\ref{sec:how-to-use} documents
how the software can be used from a user perspective;
Section~\ref{sec:implementation} is a larger part that explains some of the
most interesting details about the implementation.


\section{How to use the software}

\label{sec:how-to-use}

\subsection{Tools and setup}

The software \texttt{AtariEyes} is a Python package. It is publicly available
at the GitHub repository:
\href{https://github.com/cipollone/atarieyes}{\texttt{cipollone/atarieyes}}.
As any other Python package, it can be installed with the \texttt{pip}
command; we just need to point to this git repository. The installation
command is:
\begin{lstlisting}[style=bash]
pip install git+https://github.com/cipollone/atarieyes
\end{lstlisting}
This installs this package from the master branch. If we need to work on some
specific revision, for example on the \texttt{develop} branch, we can append
\verb!@develop! or any other commit to the previous command.

Dependencies are automatically installed by \texttt{pip}.  In Python, it is
common to run applications inside virtual environments. Just run this
installation command from a container to avoid dependency conflicts with
other applications. One rather particular dependency is TensorFlow, which is a
famous Machine Learning library that allows parallel computing. Following the
instructions of the specific container application, we can link to some
preexisting system installation if we need it. Currently, the supported
version is only 2.1, but future versions might also be compatible.

The package is written in Python~3 and the minimum version required for the
interpreter is 3.7. This requirement should be met by most modern operating
systems. If that's not the case, we suggest to use \texttt{pyenv}, which
allows environment-specific Python installations.

This procedure installs the \texttt{atarieyes} package. As we will see, we
usually use this module through its command line interface. If we want to
include some structures in future developments, we can
\lstinline[style=inlinepy]|import atarieyes|, as usual. However, for
development, it might be useful to look at the source code, by cloning it:
\begin{lstlisting}[style=bash]
git clone https://github.com/cipollone/atarieyes.git
\end{lstlisting}
This is also useful if, for any reason, some updated dependency is no longer
compatible with this package. What we can do is to \texttt{cd} to this cloned
directory, then run \texttt{poetry install}. Poetry is the container
application used for development. When we run this command, it will install
the exact dependency versions that have been used during development.


\subsection{Commands}

The package provides

Small user reference.


\section{Implementation}

\label{sec:implementation}

\subsection{\texttt{agent} Module}

\label{sec:impl-agent}

\subsubsection{\texttt{training} Module}
\subsubsection{\texttt{playing} Module}

\subsection{\texttt{streaming} Module}
\subsection{\texttt{features} Module}
\subsubsection{\texttt{models} Module}
\subsubsection{\texttt{genetic} Module}
\subsubsection{\texttt{temporal} Module}

% TODO: I've contributed to flloat

