\chapter{Introduction}

% Intro: planning and reinforcement learning
In Artificial Intelligence (AI), among the many approaches for building
intelligent agents, we can distinguish those mainly focused on knowledge and
planning, and those that mainly try out different actions to discover the
goodness of their outcomes. With the former, we refer to those developed from
\emph{classical planning}, while the latter is the recently-successful field
of \emph{reinforcement learning}. While they can be integrated, they use quite
different techniques. However, they share a common basic need: the agent must
be able to perceive meaningful events happening in the outside world.

% What is an observation in planning
In planning, in almost every practical case, there is some form of partial
observability or nondeterminism. So, agent's observations become
essential~\cite{bib:aima}. Observing, however, does not simply refers to
reading a raw input from the sensor. Instead, it means ``grounding'' all
symbols that compose the abstraction the agent adopts: essentially, all
symbols representing conditions which happen to be true in the environment,
should become true for the agent. We refer to these symbols with the term
\emph{fluents}.  Fluents are atomic propositions can change in time, whose
valuation should always reflect the current state of the world.

% Observations in RL
We could argue that reinforcement learning does not require such valuations.
Still, rewards and punishments must be somehow supplied in response to
desirable and undesirable events. We could think of providing these feedbacks
with programmed ad-hoc conditions, but this can be done just for the
simulations we create. As we will see, when the agent needs a component that
is based on logic and reasoning, we still need to valuate the truth of fluents,
even in the context of reinforcement learning.  In fact, the most successful
approaches mix these components somehow.
% TODO: cite

% Main topic and assumptions
This thesis addresses the problem of valuating fluents from complex
observations of the environment. However, this is a general topic and we'll
only work with specific classes of fluents and observations. Every choice
or assumption that restricts the applicability of this method will be pointed
out along the text. The first distinction to do is that we'll only work with
games.


\section{Reinforcement Learning in games}

% Intro to games
Games in AI are a class static environments with discrete actions. They have
always been a classic benchmark for AI, because they provide various levels of
complexity, they have few and strict rules and are easy to implement and
simulate.
% TODO: cite

% RL
Reinforcement learning (RL) is a field of AI that has shown to be successful
for many games. % TODO: cite
This is the learning method adopted here. In RL, the agent tries out
different actions and observes the reward received.  Its goal is to learn the
optimal policy, the one maximizing the cumulative reward over the whole
episode.  Most RL algorithms assume that observations and rewards can be
modelled with a Markov Decision Process (MDP). This means that: the sequence
of states create a Markov chain (the next state only depends on the previous
state and action); the rewards only depend on the current state; the
observations are an exact representation of the current state. Many learning
algorithms exist for this setting~\cite{bib:rl-book}.

% Deep RL
Neural Networks (NN) have brought new possibilities for RL: in Deep
Reinforcement Learning, the agent employs a neural network as a very
expressive approximation to the quantities it is trying to
learn~\cite{bib:deep-rl}.  The Q-value, for example, is a classic quantity in
RL that estimates the expected cumulative reward from each pair of observation
and action.  A Deep Q-Network (DQN) is able to learn this estimate for a
complex observation such as the frame of a video game, which is a
high-dimensional input~\cite{bib:atari-deeprl} that would be hard to manage
without neural networks.

% Atari games
All models and experiments in this thesis use games from the Atari~2600
collection. The framework adopted is an interface for the Atari
simulator~\cite{bib:atari-games} that maps actions to controller inputs and
returns images of the game as observations. This is almost the same
condition a human player faces when playing the same games. Other successful
works also read the current number of lives the player has. Other than that,
no internals are employed to simplify the task of the agent.

% Algorithms
The reinforcement learning algorithm adopted in this thesis is a deep variant
of Double DQN that solves few issues with simple DQN~\cite{bib:double-q}.
The motivation of this choice is that this is a relatively simple algorithm,
based on DQN, which has also proven to be successful for the specific
environments that we'll use in our experiments~\cite{bib:atari-deepq-nature}.
In fact, among Q-Network algorithms, the only ones that clearly achieve
superior performances in most games adopt a combination of all
variants~\cite{bib:rainbow}.

% There are hard games
Not all games inside the Atari~2600 collection are equally hard to solve.
Reinforcement learning agents can be trained to achieve higher performances
with respect to an average human player, mostly for environments with static
map and background, and simple strategies. Many other games, instead, require
the agent to remember previous steps and observations, for example, in
exploration tasks. One notable example is \emph{Montezuma's Revenge}, in which
agents could not improve in any way~\cite{bib:atari-deepq-nature}.
Other approaches could succeed in this game with additional information able
to guide the agent. For example, with carefully chosen initializations and
examples from human experts~\cite{bib:mz-openai-demonstrations}.

% Hard games
One issue with games like Montezuma's Revenge, is that they require long
sequences of correct actions before rewarding the agent. This is called a
\emph{sparse reward}. However, there is a more fundamental problem to be
considered first: observations and rewards, together, can't define a Markov
Decision Process, because it is essential for the agent to remember some
informations collected during the game. For example, the Montezuma's agent may
walk to the right only if it \emph{remembers} that the door in the right room
has been previously opened. This additional ability is required because of 
partial observations: a view of the current room can can't be considered a
complete state of the game, sufficient to predict future rewards.

% NMRDP
The setting just described can be modelled with a Non-Markovian Reward
Decision Process (NMRDP). Fortunately, it is possible to cast any NMRDP as a
MDP, if enough information about the history is included in each observation.
In order to render this transformation feasible, we must include as few
additional data as possible, still with an exhaustive state with respect to
the reward. As humans, we understand which sequences lead to rewards. So, an
elegant way to do this, is to declare such sequences with \emph{temporal
logics}~\cite{bib:nmrdp-logic-first}. As we will see, by tracking the
satisfaction of such temporal formula, we can provide enough information to
the agent so to employ standard algorithms developed for
MDPs~\cite{bib:degiacomo-logic-nmrdp}\cite{bib:favorito-thesis}.

% Back to fluents
This type of construction can be considered as a ``logic component'' inside
the agent, as we've previously called it. While an abstraction like this is
powerful, it is essential to correctly valuate the symbols it uses, in order
to reason about the current situation. This is a complex task for environments
with rich observations, as those allowed by Deep Reinforcement Learning.


\section{Objective of this work}

% Goal 1: computing fluents
The main purpose of this work is to devise and test a mechanism able to learn
functions which valuates the fluents we define.  Specifically, learn a
function that computes the truth value for a set of boolean conditions, given
a frame of an Atari game. Among the many different ways to accomplish this,
the most interesting techniques are those which pose the least number of
assumptions on the specific environment. In this respect, the following are
important achievements of this work to be highlighted:
\begin{itemize}
	\item Fluents are selected first. Then, the function to evaluate them is
		trained from a description of each fluent. This is harder to do than
		just training a features extractor and manually trying to associate a
		meaning to each feature.
	\item To describe the fluents we use temporal logic over finite traces such
		as \ltl{} and \ldl{}. These are employed as tools to formalize any type of
		temporal constraints the fluents are always expected to satisfy. The use
		of such logics for this purpose can be a really generic approach. This
		thesis is an initial investigation about this possibility. As a
		description of a fluent, we must consider everything that guides the
		training process. So, we will certainly consider other types of hints that
		is useful to include, such as visual hints.
	\item The training algorithm won't require any manual annotation, nor
		labelled datasets at all. The main idea is that, inside the agent, two
		components should coexist: the player and the observer. While the player 
		explores the environment, the observer can be trained from the images
		received, without further intervention.
\end{itemize}

% Goal 2: restraining bolt.
The second goal of this thesis is to demonstrate how such trained features can
be exploited by a Reinforcement Learning agent to solve hard games. Tests will
be conducted on Montezuma's Revenge, a game known to be difficult in this
class~\cite{bib:atari-deepq-nature}. In this thesis:
\begin{itemize}
		\item We provide a flexible implementation of the construction described
			in~\cite{bib:degiacomo-logic-nmrdp}\cite{bib:favorito-thesis}, for
			temporal goals.
		\item A deep agent architecture is proposed to merge the technique above
			for the Deep Reinforcement Learning case.
		\item This implementation is then used to specify a temporal goal in
			\ldl{}, sufficient to guide the agent through hard environments.
\end{itemize}

\section{Structure of the thesis}

% TODO
