\chapter{Introduction}

% Intro logic
A classic and important branch of Artificial Intelligence (AI) aims at
developing agents that select their actions through some form of logic
reasoning, such as planning. One of the main advantages of these approaches is
that reasoning proceeds by manipulating \emph{abstractions}. In fact, in
logic, we can define symbols that represent any meaningful event or condition
that should be considered. For example, some propositional symbols might
represent conditions such as ``the door is closed'' or ``I am holding an
object''. We also call these propositional symbols with the term
``\emph{fluents}'', because their truth can change over time.

% Grounding
However, these methods imply one fundamental ability: at each instant, the
agent must be able to decide whether those propositions are true. This means
that all symbols that represent conditions which happen to be true in the
environment, must be true for the agent. This ``grounding'' process can be
really hard in complex environments, because the agent's sensors may return a
noisy and multidimensional output, difficult to interpret.

% Observations in RL
Reinforcement Learning (RL) is a recently-successful field of AI, that
proceeds by learning a policy that maximize the rewards received.  We could
argue that RL does not require the valuations just mentioned. Still, rewards
and punishments must be somehow supplied in response to desirable and
undesirable events. We could think of providing these feedbacks
with programmed ad-hoc conditions, but this can be easily done just for the
simulations we create. Furthermore, as we will see, some complex tasks can
only be solved through a combination of RL and logic-based methods; thus,
introducing all the needs of the latter.

% Bridge: goal
With this thesis, we defined and implemented an agent based on temporal logics
and Deep Reinforcement Learning. This also required to investigate new ways to
solve the fluents valuation problem that has been just described.
% Structure
Section~\ref{sec:intro-related} shows the general context and the works this
thesis is related to. In Section~\ref{sec:intro-objective}, the goal and the
achievements of this work will be described more precisely.
% TODO: move
%% Main topic and assumptions
%This thesis addresses the problem of valuating fluents from complex
%observations of the environment. However, this is a general topic and we'll
%only work with specific classes of fluents and observations. Every choice
%or assumption that restricts the applicability of this method will be pointed
%out along the text. The first distinction to do is that we'll only work with
%games.


\section{Related works}

\label{sec:intro-related}

% RL
Reinforcement Learning (RL) is an area of Machine Learning in which the agent
is trained by receiving rewards and punishments in response to its actions.
This technique can be also used unknown environments (where a model of the
dynamics is not available), because the agent learns by trying all actions and
by remembering those that lead to the highest rewards. As we will see in
Chapter~\ref{ch:rl}, most RL algorithms assume that the environment can be
modelled with a Markov Decision Process. Many learning algorithms exist in
this setting~\cite{bib:rl-book}.

% Deep RL
Neural Networks (NN) have brought new possibilities for RL: in Deep
Reinforcement Learning, the agent employs a neural network as a very
expressive function approximator for the quantities it is trying to
learn~\cite{bib:deep-rl}. For example, the optimal q-value is an important
quantity in RL, that the agents are usually designed to learn from the
observations received. The Deep Q-Network (DQN)
algorithm~\cite{bib:atari-deeprl} is one the first to successfully employ
neural networks in RL. They have shown that a Deep RL agent can be trained
directly from complex observations such as the frames of a video game. Without
modifications, the same agent has been able to reach human-level performances
in many games.

% Atari games
Games have always been a classic benchmark for AI algorithms, because they
provide various levels of complexity, they have few and strict rules, and they
are easy to implement and simulate. Regarding Deep RL, many authors have
tested their algorithms on the collection of video games
``Atari~2600''~\cite{bib:atari-games}. In this thesis, we'll use and
experiment with the same environments.

% Algorithms
The reinforcement learning algorithm adopted in this thesis is called Double
DQN~\cite{bib:double-q}. The motivation of this choice is that this is a
relatively simple algorithm, based on DQN, which has also proven to be
successful for the specific environments that we'll use in our
experiments~\cite{bib:atari-deepq-nature}.  In fact, among Q-Network
algorithms, the only ones that clearly achieve superior performances in most
games adopt a combination of all DQN variants~\cite{bib:rainbow}.

% Some games are hard
If we look at the results in~\cite{bib:atari-deepq-nature}, DQN agents are
able to learn excellent policies for many games. However, for many other
environments of the same collection, the agents struggle to learn and, in some
cases, they don't learn anything at all. The worst performances have been
measured for the \emph{Montezuma's Revenge} environment. In this game, the
only methods that were able to achieve good policies adopted a combination of
expert imitation and manual restarts~\cite{bib:mz-openai-demonstrations}.
In Section~\ref{sec:non-markov}, we'll investigate the main cause of these
difficulties.

% Bridge to rewarding behaviours
As we will see throughout this thesis, a promising solution for these
environments is the construction provided in~\cite{bib:degiacomo-logic-nmrdp}
and~\cite{bib:bolt}. The former work~\cite{bib:degiacomo-logic-nmrdp} has
shown that a Non-Markovian Reward Decision Process (NMRDP) can be easily
declared with linear-time temporal logics, and it has provided a translation
from this NMRDP to a classic MDP. The logics adopted are~\ltl{} and~\ldl{}.
This idea was initially introduced in~\cite{bib:nmrdp-logic-first} for a
linear temporal logic of the past. The latter work~\cite{bib:bolt}, instead,
has shown that through same construction, temporal logics can be used to
provide a RL agent with additional rewards, thus influencing its final
behaviour. This paper named this additional module with ``Restraining Bolt''.
Thorough this thesis, it might be handy to use this name to refer to this
logic construction.


\section{Objective and results}

\label{sec:intro-objective}

% TODO: separate high-level objective and specific list of results.

% Goal 1: computing fluents
The main purpose of this work is to devise and test a mechanism able to learn
functions which valuate the fluents we define.  Specifically, learn a
function that computes the truth value for a set of boolean conditions, given
a frame of an Atari game. Among the many different ways to accomplish this,
the most interesting techniques are those which pose the least number of
assumptions on the specific environment. In this respect, the following are
important achievements of this work to be highlighted:
\begin{itemize}
	\item Fluents are selected first. Then, the function to evaluate them is
		trained from a description of each fluent. This is harder to do than
		just training a features extractor and manually trying to associate a
		meaning to each feature.
	\item To describe the fluents we use temporal logic over finite traces such
		as \ltl{} and \ldl{}. These are employed as tools to formalize any type of
		temporal constraints the fluents are always expected to satisfy. The use
		of such logics for this purpose can be a really generic approach. This
		thesis is an initial investigation about this possibility. As a
		description of a fluent, we must consider everything that guides the
		training process. So, we will certainly consider other types of hints that
		is useful to include, such as visual hints.
	\item The training algorithm won't require any manual annotation, nor
		labelled datasets at all. The main idea is that, inside the agent, two
		components should coexist: the player and the observer. While the player 
		explores the environment, the observer can be trained from the images
		received, without further intervention.
\end{itemize}

% Goal 2: restraining bolt.
The second goal of this thesis is to demonstrate how such trained features can
be exploited by a Reinforcement Learning agent to solve hard games. Tests will
be conducted on Montezuma's Revenge, a game known to be difficult in this
class~\cite{bib:atari-deepq-nature}. In this thesis:
\begin{itemize}
		\item We provide a flexible implementation of the construction described
			in~\cite{bib:degiacomo-logic-nmrdp}\cite{bib:favorito-thesis}, for
			temporal goals.
		\item A deep agent architecture is proposed to merge the technique above
			for the Deep Reinforcement Learning case.
		\item This implementation is then used to specify a temporal goal in
			\ldl{}, sufficient to guide the agent through hard environments.
\end{itemize}
% TODO: I also contributed to flloat


\section{Structure of the thesis}

The rest of this thesis is structured as follows:
\begin{description}[style=nextline]
	\item[\ref{ch:logics}~--~\nameref{ch:logics}]
		In this chapter, an important formalism that will be used throughout the
		thesis is reviewed. We introduce the reader to concepts such as fluents,
		traces and linear-time temporal logics. Then, we will define the Linear
		Dynamic Logic, which is the specific temporal logic used in this text.
	\item[\ref{ch:rl}~--~\nameref{ch:rl}]
		This chapter presents the second large background of this thesis.  We will
		see Reinforcement Learning from the basic concepts and assumptions, in
		Section~\ref{sec:rl}. Section~\ref{sec:deep-rl} reviews some of the
		advancements of the Deep RL field of last years. Then, in
		Section~\ref{sec:non-markov}, we will analyze what happens when the most
		common assumptions of RL (and of Deep RL) are falsified.
		Section~\ref{sec:rb} presents a recent work that allows to elegantly solve
		this issue thanks to temporal logics.
	\item[\ref{ch:fluents}~--~\nameref{ch:fluents}]
		In this and the following chapters, we will see the innovative
		contributions of this thesis. Here, we'll see how the agent can be trained
		to valuate a class of fluents to their expected truth. A model for the
		valuation function will be proposed and a training algorithm.
	\item[\ref{ch:atarieyes}~--~\nameref{ch:atarieyes}]
		This chapter presents the software that implemented the concepts presented
		in the previous chapters. I will be first explained from a use
		perspective, then the most interesting implementation details will follow.
	\item[\ref{ch:experiments}~--~\nameref{ch:experiments}]
		This chapter contains experiments and training outcomes for two Atari
		games. Experiments will be finalized to test the effectiveness of learning
		the fluents valuation functions and the capabilities of the ``restrained''
		Deep RL agents.
	\item[\ref{ch:conclusions}~--~\nameref{ch:conclusions}]
		This thesis ends with some final considerations about: the main
		conclusions that can be derived from this work; the strength of this
		approach and its weakness; and its possibilities for improvement.
\end{description}

% TODO: missing something: Deep RL agent with RB
